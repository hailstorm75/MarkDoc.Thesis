%%% Fiktivní kapitola s ukázkami sazby

\chapter{Nápověda k~sazbě}

\section{Úprava práce}

Vlastní text práce je uspořádaný hierarchicky do kapitol a podkapitol,
každá kapitola začíná na nové straně. Text je zarovnán do bloku. Nový odstavec
se obvykle odděluje malou vertikální mezerou a odsazením prvního řádku. Grafická
úprava má být v~celém textu jednotná.

Zkratky použité v textu musí být vysvětleny vždy u prvního výskytu zkratky (v~závorce nebo
v poznámce pod čarou, jde-li o složitější vysvětlení pojmu či zkratky). Pokud je zkratek
více, připojuje se seznam použitých zkratek, včetně jejich vysvětlení a/nebo odkazů
na definici.

Delší převzatý text jiného autora je nutné vymezit uvozovkami nebo jinak vyznačit a řádně
citovat.

% \begin{figure}
%     \centering
%     \caption{What output formats do you wish to have for your documentation?}
%     \begin{tikzpicture}
%         \begin{axis}
%         [
%             ybar,
%             enlargelimits=0.15,
%             ylabel={\#Average Marks}, % the ylabel must precede a # symbol.
%             xlabel={\ Students Name},
%             symbolic x coords={Static HTML, PDF, Markdown, XML, \LaTeX, Other}, % these are the specification of coordinates on the x-axis.  
%             xtick=data,
%             nodes near coords, % this command is used to mention the y-axis points on the top of the particular bar.  
%             nodes near coords align={vertical},
%             ]
%         \addplot coordinates {(Static HTML,16) (PDF,7) (Markdown,6) (XML,6) (\LaTeX,1) (Other,3) };

%         \end{axis}
%     \end{tikzpicture}
% \end{figure}

\begin{figure}
    \centering
    \caption{Would you integrate such a tool in your CI/CD process?}
    \begin{tikzpicture}
        \pie[sum=auto , after number=]{9/Yes, 2/No, 9/Maybe}
    \end{tikzpicture}
\end{figure}

\section{Jednoduché příklady}

Mezi číslo a jednotku patří úzká mezera: šířka stránky A4 činí $210\,\rm mm$, což si
pamatuje pouze $5\,\%$ autorů. Pokud ale údaj slouží jako přívlastek, mezeru vynecháváme:
$25\rm mm$ okraj, $95\%$ interval spolehlivosti.

Rozlišujeme různé druhy pomlček:
červeno-černý (krátká pomlčka),
strana 16--22 (střední),
$45-44$ (matematické minus),
a~toto je --- jak se asi dalo čekat --- vložená věta ohraničená dlouhými pomlčkami.

V~českém textu se používají \uv{české} uvozovky, nikoliv ``anglické''.

% V tomto odstavci se vlnka zviditelňuje
{
\def~{{\tt\char126}}
Na některých místech je potřeba zabránit lámání řádku (v~\TeX{}u značíme vlnovkou):
u~před\-lo\-žek (neslabičnych, nebo obecně jednopísmenných), vrchol~$v$, před $k$~kroky,
a~proto, \dots{} obecně kdekoliv, kde by při rozlomení čtenář \uv{ško\-brt\-nul}.
}
