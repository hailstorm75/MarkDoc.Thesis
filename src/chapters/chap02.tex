\chapter{Documentation tools}
\section{What is available?}
The current documentation tool market provides the following solutions:

\begin{enumerate}
    \item \href{https://github.com/dotnet/docfx}{DocFX}
    \item \href{https://www.doxygen.nl/}{Doxygen}
    \item \href{https://github.com/KirillOsenkov/SourceBrowser}{Source Browser}
    \item \href{https://github.com/lijunle/Vsxmd}{Vsxmd}
    \item \href{https://github.com/discosultan/vsdoc-2-md}{vsdoc-2-md}
\end{enumerate}

What follows are evaluations of each tool. This helped get a better understanding of the current offering and gain prospect on features and improvements that the custom tool can take advantage of.

\subsection{DocFX}

\textit{DocFX} is an open-source documentation generation tool developed by Microsoft that not only supports .NET languages C\#, F\#, and Visual Basic, but also Java, JavaScript, TypeScript, Python, and REST. Additionally, it can use raw Markdown files as input.

However, the \textit{DocFX} only outputs static \ref{itm:html} pages. The only available customizability is via templates for said static pages.

The tool is capable and is the go-to solution for .NET projects, but it doesn't solve the task of outputting Markdown files.

\subsection{Doxygen}

\textit{Doxygen} is the industry-standard documentation generating tool originally made for C++ source code documentation. Nevertheless, it has over time added support for many popular programming languages such as C, C\#, Java, Python, and many more.

The tool provides an extensive set of supported output formats:
\begin{itemize}
    \item Static \ref{itm:html}
    \item \LaTeX\footnote{Document preparation system}
    \item Man pages\footnote{User manual type that is part of Unix operating systems \cite{credocs_limited_latex_2022}}
    \item \ref{itm:rtf}
    \item \ref{itm:xml}
\end{itemize}

\subsection{Source Browser}

\textit{Source Browser} is a tool that generates a website for browsing source code and its documentation. The tool is used by Microsoft, for example, to allow developers to browse the source code of .NET.

The generated output is not fully static and has to be hosted on an ASP.NET Core website to support searching.

\subsection{Vsxmd} \label{ssec:vsxmd}

\textit{Vsxmd} generates a single Markdown file for all types in a given assembly. Moreover, the tool has no \ref{itm:ui} and works as a NuGet\footnote{NuGet is the package manager for .NET \cite{microsoft_nuget_nodate}} package, that is added to the project designated for documentation generation. Thus, configuring this tool is done via .NET project settings.

It is not possible to navigate the documentation, as no links are generated.

\subsection{Vsdoc-2-md}

\textit{Vsdoc-2-md} is an entirely unusable tool, as it is purely web-based and generates documentation from the provided \ref{itm:xml} documentation source file. The tool is limited to processing only one file at a time.

Just like \textit{\nameref{ssec:vsxmd}}, no links are generated; thus, it is not possible to navigate the documentation.