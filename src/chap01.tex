\chapter{Documentation tools}
\section{What is available?}
The current documentation tool market provides the following solutions:

\begin{enumerate}
    \item \label{itm:firstTool} \href{https://github.com/dotnet/docfx}{DocFX}
    \item \label{itm:secondTool} \href{https://www.doxygen.nl/}{Doxygen}
    \item \label{itm:thirdTool} \href{https://github.com/KirillOsenkov/SourceBrowser}{Source Browser}
    \item \label{itm:fourthTool} \href{https://github.com/lijunle/Vsxmd}{Vsxmd}
    \item \label{itm:fifthTool} \href{https://github.com/discosultan/vsdoc-2-md}{vsdoc-2-md}
\end{enumerate}

\subsection{DocFX}

DocFX is an open-source documentation generation tool developed by Microsoft that not only supports .NET languages C\#, F\#, and Visual Basic, but also Java, JavaScript, TypeScript, Python and REST. Additionally, it can use raw Markdown files as input.

However, the DocFX only outputs static HTML\ref{itm:html} pages. The only available customizability is via templates for said static pages.

The tool is definitely powerful and is the go to solution for .NET projects, but it doesn't solve the task of outputting Markdown files.

\subsection{Doxygen}

Doxygen is industry standard documentation generation tool originally made for C++ source code documentation. Nevertheless, the tool has added support for many popular programming languages including C, C\#, Java, Python, and many more.

The tool provides an extensive set of supported output formats:
\begin{itemize}
    \item Static \ref{itm:html}
    \item \LaTeX\footnote{Document preparation system}
    \item Man pages\footnote{User manual type that is part of Unix operating systems\cite{credocs_limited_latex_2022}}
    \item \ref{itm:rtf}
    \item \ref{itm:xml}
\end{itemize}

\subsection{Source Browser}

\subsection{Vsxmd}

Vsxmd (\ref{itm:fourthTool})) generates a single Markdown file for all types in a given assembly. Moreover, the tool has no UI and works as a NuGet\footnote{NuGet is the package manager for .NET \cite{microsoft_nuget_nodate}} package, that is added to the project designated for documentation generation. Thus, configuring this tool is done via .NET project settings.

\subsection{Vsdoc-2-md}

On the other hand, vsdoc-2-md (\ref{itm:fifthTool}) is an entirely unusable tool, as it is purely web-based and generates documentation from the provided \ref{itm:xml} documentation source file. The tool is limited to processing only one file at a time.