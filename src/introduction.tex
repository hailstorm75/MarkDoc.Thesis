\chapter*{Introduction}
\addcontentsline{toc}{chapter}{Introduction}

\section*{Goal}

Source code documentation is an essential part of any quality project. Writing, and keeping such documentation up to date is often overlooked, or too cumbersome to do, due to strict deadlines. This might lead to the degradation of project quality, as developers who leave said projects usually take their know-how with them without properly passing them down to their replacements.

In the rare cases where source code documentation is taken seriously, it is also considered good practice to extract said documentation from the source code into a searchable, public format such as \ref{itm:html}. This is accomplished with documentation generating tools.

For Microsoft \ref{gloss:dotnetlabel} projects there is a selection of documentation generating tools available, which includes: DocFX, Doxygen, SourceBrowser, and others. These solutions primarily generate output in \ref{itm:html}, which is sufficient for the majority of projects. However, most of them lack in support for other output formats like PDF, Markdown, or custom formats. Moreover, customization of these tools is very limited, which handicaps projects from using said documentation generating tools fit their exact needs. More often than not, these tools lack in \ref{itm:ui}/\ref{itm:ux}, introducing an unnecessary learning curve.

With that in mind, there is visible room for improvement. A desirable documentation generating tool would be easy to use, modern, extensible, and rich in output format support.

Thus, the goal of this thesis is to satisfy this desire by creating a custom documentation generating tool. Specifically for \ref{gloss:dotnetlabel} projects, that will focus on extensibility, ease of use, and rich support of output formats.

\section*{Motivation}

The motivation for creating a custom tool is primarily a personal need to provide easy access to code documentation. Since most open source projects are hosted either on GitHub.com\footnote{A Microsoft-owned Git hosting platform} or GitLab.com\footnote{An independent Git hosting platform}, it makes sense to utilize said platforms built-in wiki pages for hosting source code documentation for consistency. A minority of developers use Bitbucket\footnote{TODO} to host their public open source projects as this platform is predominantly used in the enterprise market. Additionally, corporate clients who purchase Bitbucket usually purchase Confluence alongside it, which serves as a documentation hosting platform. With this in mind, directly supporting Bitbuckets wiki is not a priority; however, adding future support for confluence is a potential business opportunity.

When attempting to find tools that generate documentation, none had the desired extensibility and were mainly limited to only creating static \ref{itm:html} pages
\footnote{A static web page is a page that is built using \ref{itm:html} code and features the same presentation and content, regardless of user identity or other factors \cite{techopedia_what_2017}}.
Since \ref{gloss:git} hosting platform wiki pages predominantly utilize \ref{gloss:markdown} for displaying formatted text\footnote{Apart from \ref{gloss:markdown}, said \ref{gloss:git} platforms support more formats; however, the latter has the richest formatting capabilities}, static \ref{itm:html} pages are out of the question.

And so, the idea to develop a custom tool that would create \ref{gloss:markdown} documentation from \ref{gloss:dotnetlabel} libraries for \ref{gloss:git} platforms came to fruition. However, focusing only on one output format would be a waste of time and effort, as few to none would need such a tool. Thus, the result of the development should be a generic tool that allows anyone to modify it to output to any desirable format.

\section*{Strategy}
asdf
