\chapter*{Introduction}
\addcontentsline{toc}{chapter}{Introduction}

The motivation for creating a custom tool for generating documentation from \ref{gloss:dotnetlabel} libraries is a personal need to provide easy access to code documentation. Since most open source projects are hosted either on GitHub.com\footnote{A Microsoft-owned Git hosting platform} or GitLab.com\footnote{An independent Git hosting platform}, it makes sense to utilize said platforms built-in wiki pages for hosting source code documentation for consistency. A minority of developers use Bitbucket\footnote{TODO} to host their public open source projects as this platform is predominantly used in the enterprise market. Additionally, corporate clients who purchase Bitbucket usually purchase Confluence alongside it, which serves as a documentation hosting platform. With this in mind, directly supporting Bitbuckets wiki is not a priority; however, adding future support for confluence is a potencial business opportunity.

When attempting to find tools that generate documentation, none had the desired extensibility and were mainly limited to only creating static \ref{itm:html} pages
\footnote{A static web page is a page that is built using \ref{itm:html} code and features the same presentation and content, regardless of user identity or other factors \cite{techopedia_what_2017}}.
Since \ref{gloss:git} hosting platform wiki pages predominantly utilize \ref{gloss:markdown} for displaying formatted text\footnote{Apart from \ref{gloss:markdown}, said \ref{gloss:git} platforms support more formata; however, the latter has the richest formatting cappabilities}, static \ref{itm:html} pages are out of the question.

And so, the idea to develop a custom tool that would create \ref{gloss:markdown} documentation from \ref{gloss:dotnetlabel} libraries for \ref{gloss:git} platforms came to fruition. However, focusing only on one output format would be a waste of time and effort, as few to none would need such a tool. Thus, the result of the development should be a generic tool that allows anyone to modify it to output to any desirable format.

Accomplishing this task will require analyzing the currently available documentation tools, getting user feedback, planning out the project structure, and finally developing the project itself.

